% Bestätigung über Sachzuwendungen, 2014
% Uwe Ziegenhagen, ziegenhagen@gmail.com
% 

\documentclass[11pt,ngerman]{scrartcl}
\usepackage[utf8]{inputenc}
\usepackage[T1]{fontenc}
\usepackage{booktabs}
\usepackage{babel}
\usepackage{graphicx,dashrule}
\usepackage{csquotes}
\usepackage{paralist}
\usepackage{mdframed}
\usepackage{wasysym}
\usepackage{tabu}
\usepackage{ifthen}

\newcommand{\VAR}[1]{#1}
\newcommand{\BLOCK}[1]{#1}

\newboolean{sammel} % Sammelbestätigung
\setboolean{sammel}{true}

\renewcommand{\familydefault}{\sfdefault}
\RequirePackage[scaled=0.9]{helvet}
\pagestyle{empty}

\newcommand{\mychecked}{\scalebox{1.5}{\XBox}~} % \mycheckedBox
\newcommand{\unmychecked}{\scalebox{1.5}{\Square}}
\usepackage[]{eurosym}

\usepackage{xcolor}
\usepackage[a4paper,left=2cm,right=2cm,top=1cm,bottom=1cm]{geometry}
\setlength{\parindent}{0pt}
\setlength{\parskip}{0pt}


\mdfdefinestyle{MyFormStyle}{%
    linewidth=1pt,
    skipbelow=\topskip,
    skipabove=\topskip
}

\newcommand{\MyForm}[2][1.0cm]{%
    \begin{mdframed}[style=MyFormStyle]%
    {\noindent\footnotesize#2}\vspace{#1}%
    \end{mdframed}%
}

\newcommand{\MyFormBox}[3][1.0cm]{%
    \begin{mdframed}[style=MyFormStyle]%
    {\noindent\footnotesize#2 \vspace*{1em} \par\normalsize #3}\vspace*{#1}%
    \end{mdframed}%
}

\begin{document}
\MyFormBox[0.0cm]{Aussteller (Bezeichnung und Anschrift der steuerbegünstigten Einrichtung)}{Musterverein Köln e.\,V. \\ Mustergasse 1 \\ 123456~Köln}

{\bfseries\large \ifthenelse{\boolean{sammel}}{Sammelbestätigung}{Bestätigung} über Geldzuwendungen/Mitgliedsbeiträge}\vspace*{1em}

im Sinne des § 10b des Einkommensteuergesetzes an eine der in § 5 Abs. 1 Nr. 9 des Körperschaftsteuergesetzes bezeichneten Körperschaften, Personenvereinigungen oder Vermögensmassen 

\MyFormBox[0.0cm]{Name und Anschrift des Zuwendenden}{Maria Musterfrau, Musterallee 123, 12345 Köln}

\begin{tabu}{|[1pt]p{0.3\textwidth}|[1pt]p{0.32\textwidth} |[1pt]p{0.3\textwidth}|[1pt]} \tabucline[1pt]{-}
\scriptsize \ifthenelse{\boolean{sammel}}{Gesamtbetrag}{Betrag} der Zuwendung - in Ziffern - & \scriptsize- in Buchstaben - & \scriptsize \ifthenelse{\boolean{sammel}}{Zeitraum der Sammelbestätigung:}{Tag der Zuwendung} \\ 
\vspace*{1em} & & \\ 
48,00 EUR  & xxx-Vier-Acht-xxx & 01.01.2019 -- 31.12.2019 \\
\vspace*{1em} & & \\ \tabucline[1pt]{-}
\end{tabu}

\ifthenelse{\boolean{sammel}}{}{\vspace*{0.5em}Es handelt sich um den Verzicht auf Erstattung von Aufwendungen: 	\hspace{1em}	Ja  \mychecked	\hspace{1em}	Nein  \unmychecked}

\vspace*{1.5em}
\mychecked\,Wir sind wegen Förderung von Nr.\,1 (Förderung von Wissenschaft und Forschung) und  Nr.\,7 (Förderung der Erziehung, Volks- und Berufsbildung sowie der Studentenhilfe) des §52 AO nach dem Freistellungsbescheid bzw. nach der Anlage 
zum Körperschaftssteuerbescheid des Finanzamts Köln StNr 123/4567/7890, vom 12.01.2010 für den letzten Veranlagungszeitraum 2015--2017 nach §~5 Abs. 1 Nr.~9 des Körperschaftssteuergesetzes von der Körperschaftssteuer und nach §~3 Nr.~6 des Gewerbesteuergesetzes von der Gewerbesteuer befreit. \vspace{1em}

\mychecked\,Die Einhaltung der satzungsmäßigen Voraussetzungen nach den §§ 51, 59, 60 und 61 AO wurde vom Finanzamt Finanzamt Köln, StNr. 123/4567/7890, mit Bescheid vom 12.01.2010 nach § 60 AO gesondert festgestellt. Wir fördern nach unserer Satzung Nr. 1 (Förderung von Wissenschaft und Forschung) und und Nr. 7 (Förderung der Erziehung, Volks- und Berufsbildung sowie der Studentenhilfe) des §52 Abs. 2 Satz 1 Nr. 1 und 7 AO.
 
\begin{mdframed}[style=MyFormStyle]%
\footnotesize Es wird bestätigt, dass die Zuwendung nur zur Förderung von Nr.\,1 (Förderung von Wissenschaft und Forschung) und  Nr.\,7 (Förderung der Erziehung, Volks- und Berufsbildung einschließlich der Studentenhilfe) des §52 AO verwendet wird. \vspace{0.5em}

\textbf{Nur für steuerbegünstigte Einrichtungen, bei denen die Mitgliedsbeiträge steuerlich nicht absetzbar sind:} \\
\unmychecked\, Es wird bestätigt, dass es sich nicht um einen Mitgliedsbeitrag handelt, dessen Abzug nach §\,10 Abs. 1 des Einkommensteuergesetzes ausgeschlossen ist. 

\end{mdframed} 

\ifthenelse{\boolean{sammel}}{\vspace*{0.5em}Es wird bestätigt, dass über die in der Gesamtsumme enthaltenen Zuwendungen keine weiteren Bestätigungen, weder formelle Zuwendungsbestätigungen noch Beitragsquittungen o.ä., ausgestellt wurden und werden. 

\vspace*{0.5em}Ob es sich um den Verzicht auf Erstattung von Aufwendungen handelt, ist der Anlage zur Sammelbestätigung zu entnehmen.}{}

\vspace*{2.5em} 

Köln, den \today \hspace*{20em} Max Schatzmaster

\hrule

\vspace*{0.5em} (Ort, Datum und Unterschrift des Zuwendungsempfängers) 

\paragraph{Hinweis:} Wer vorsätzlich oder grob fahrlässig eine unrichtige Zuwendungsbestätigung erstellt oder veranlasst, dass 
Zuwendungen nicht zu den in der Zuwendungsbestätigung angegebenen steuerbegünstigten Zwecken verwendet 
werden, haftet für die entgangene Steuer (§ 10b Abs. 4 EStG, § 9 Abs. 3 KStG, § 9 Nr. 5 GewStG). 

Diese Bestätigung wird nicht als Nachweis für die steuerliche Berücksichtigung der Zuwendung anerkannt, wenn das Datum des Freistellungsbescheides länger als 5 Jahre bzw. das Datum der Feststellung der Einhaltung der satzungsmäßigen Voraussetzungen nach § 60 Abs. 1 AO länger als 3 Jahre seit Ausstellung des Bescheides zurückliegt (§63 Abs. 5 AO). 

\vfill \footnotesize{\textcolor{gray}{Mitgliedsnummer 2}}

\clearpage

{\bfseries\large Anlage zur Sammelbestätigung} \vspace*{2em}


\begin{tabular}{lllr}
\toprule
Buchungstag & Vorgang &              Kategorie &  Betrag \\
\midrule
 02-01-2019 &    7451 &  Mitgliedsbeitrag\_2110 &   25.00 \\
 03-02-2019 &    7452 &  Mitgliedsbeitrag\_2110 &   23.00 \\
\bottomrule
\end{tabular}


\end{document}